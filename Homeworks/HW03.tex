\documentclass[12pt]{article}
\pagestyle{empty}
\usepackage{amsmath, amssymb, amsthm}
\usepackage{latexsym, epsfig, ulem, cancel, multicol, hyperref}
\usepackage{graphicx, tikz, subfigure,pgfplots}
\usepackage[margin=1in]{geometry}
\setlength{\parindent}{0pt}
\usepackage{multirow}
\usepackage{mathtools}


\usepackage{verbatim}
\usepackage{tikz}
\usepackage{pgfplots}
\newcommand{\T}[0]{\top}
\newcommand{\F}[0]{\bot}
\newcommand{\liminfty}[1]{\lim_{#1 \to \infty}}
\newcommand{\limzero}[1]{\lim_{#1 \to 0}}
\newcommand{\Z}{\mathbb{Z}}
\newcommand{\R}{\mathbb{R}}
\newcommand{\C}{\mathbb{C}}
\newcommand{\Q}{\mathbb{Q}}
\newcommand{\odd}[0]{\mathbb{Z} - 2\mathbb{Z}}
\newcommand{\lineint}[1]{\int_{#1}}
\newcommand{\pypx}[2]{\frac{\partial #1}{\partial #2}}
\newcommand{\divg}{\nabla \cdot}
\newcommand{\curl}{\nabla \times}
\newcommand{\dydx}[2]{\frac{d #1}{d #2}}
\newcommand{\sqbkt}[1]{\left[ #1 \right]}
\newcommand{\paren}[1]{\left( #1 \right)}
\newcommand{\tribkt}[1]{\left< #1 \right>}
\newcommand{\abso}[1]{\left|#1 \right|}
\newcommand{\zero}{\{0\}}
\newcommand{\then}{\rightarrow}
\newcommand{\nonneg}{\Z^+ \cup \{0\}}

\newcommand{\defcomp}{\exists r,s\in \Z^+ \paren{n=rs \wedge \paren{1<r<n} \wedge \paren{1<s<n}}}
\newcommand{\defprime}{\forall r,s \in \Z ^+ \paren{n=rs \rightarrow \paren{r = 1 \wedge s = n}\veebar \paren{r=n \wedge s=1}}}

\newcommand{\wsnumber}{1}
\newcommand{\wstopic}{Vectors}
\pgfplotsset{
    every linear axis/.append style={
       axis x line=center,
       axis y line=center,
       xlabel={$x$},
       ylabel={$y$}
    },
    every axis plot/.append style={thick,mark=none}
}
\tikzset{
    point/.style={circle,draw,fill,minimum width=0.3ex,inner sep=0pt,outer sep=0pt},
    every label/.append style={black}
}


\usepackage[margin=1in]{geometry}
\usepackage{amsmath, amssymb, amsthm, graphicx, hyperref}
\usepackage{enumerate}
\usepackage{fancyhdr}
\usepackage{multirow, multicol}
\usepackage{tikz}
\pagestyle{fancy}
\fancyhead[RO]{Dennis Li}
\fancyhead[LO]{Summer 6W2 2024 MA-UY 2314}
\usepackage{comment}
\newif\ifshow
\showfalse

\ifshow
  \newenvironment{solution}{\textbf{Solution.}}{}
\else
  \excludecomment{solution}
\fi

\renewcommand{\thefootnote}{\fnsymbol{footnote}}
\usepackage{comment}


\newtheorem*{remark}{Remark}


\begin{document}

\begin{center}
\ifshow
  \textbf{\Large Homework 2 Solution}\\
\else
  \textbf{\Large Homework 3}\\
\fi
Due: Tuesday July 23\\via Gradescope\\
\end{center}

\hrule

\vspace{0.2cm}

\begin{enumerate}[$\bullet$]  
\item Late homework is not accepted.  Lateness due to technical issues will not be excused.  
\end{enumerate}

\hrule

\vspace{0.5cm}


\begin{enumerate}
    \item Section 4.3 \#14, 18
        \begin{enumerate}
            \item[14.] The cube of any rational number is a rational number.
            \[
            \forall n \in \Q \paren{n^3 \in \Q}
            \]
            \begin{proof}[proof 4.3.14]:\\
                let $n \in \Q$, then for some $m \in \Z, \; n \in \Z - \zero$, $n$ can be written as
                \[
                n = \frac{m}{n}
                \]
                and we have 
                \[
                n^3 = \frac{m^3}{n^3}
                \]
                notice that $m^3 \in \Z, \; n^3 \in \Z - \zero$ since integer is closed under multiplication, therefore $n^3 \in \Q$
                
            \end{proof}
            
            \item[18.]If \( r \) and \( s \) are any two rational numbers, then \( \frac{r + s}{2} \) is rational.
                \[
                \forall r, s \in \Q \paren{\frac{r + s}{2} \in \Q}
                \]
                \begin{proof}[proof 4.3.18]:\\
                    let $r,s \in \Q$, then for some $a,c \in \Z, \; b,d \in \Z - \zero$, $r,s$ can be written as
                    \[
                    r = \frac{a}{b} \quad s = \frac{c}{d}
                    \]
                    and by algebra $r+s$ can be written as 
                    \[
                    \frac{ad+bc}{bd}
                    \]
                    divide it by 2 to obtain $\frac{r+s}{2}$
                    \[
                    \frac{ad+bc}{2bd}
                    \]
                    notice $ad+bc \in \Z$ and $bd \in \Z - \zero$, therefore $\frac{r+s}{2}\in \Q$

                \end{proof}
                
        \end{enumerate}

        \newpage

        \item Section 4.3 \# 32, 33, 36
            \begin{enumerate}
                \item[32.] Prove that for every real number $c$, if $c$ is a root of a polynomial with rational coefficients, then $c$ is a root of a polynomial with integer coefficients. 
                \begin{proof}
                    let $c \in \R$, let $p_0(x)$ be a polynomial in the following form
                    \[
                    p_0(x) = a_ix^i + a_{i-1}x^{i-1} \ldots a_2x^2 + a_1x +a_0
                    \]
                    for some $a_i \in \Q$ for some $i \in \Z^+ \cup \zero$ degree, since $a_i$ is rational, it can be written in the following form
                    \[
                    a_i =\frac{m_i}{n_i}
                    \]
                    for some $m_i \in \Z$ and for some $n_i \in \Z - \zero$, and the polynomial can be rewritten as 
                    \[
                    p_0(x) = \frac{m_i}{n_i}x^i + \frac{m_{i-1}}{n_{i-1}}x^{i-1} \ldots \frac{m_2}{n_2}x^2 + \frac{m_1}{n_1}x + \frac{m_0}{n_0}
                    \]
                    Define a common denominator for all coefficient to be
                    \[
                    \mathbf{d} = \prod_{k=0}^{i}n_k
                    \]
                    where $\mathbf{d} \in \Z$ since integers are closed under multiplication, therefore the polynomial can be rewritten as
                    \[
                    p_(x) = \mathbf{d}p_0(x)
                    \]
                    and $p_1(x)$ can be written as 
                    \[
                    p_1(x) = k_ix^i + k_{i-1}x^{i-1} \ldots k_2x^2 + k_1x +k_0
                    \]
                    where $k_i$ is defined to be
                    \[
                    k_i = \frac{\mathbf{d}}{n_i}
                    \]
                    since $\mathbf{d}$ is defined to be the product of all $n_i$, therefore $k_i \mid \mathbf{d}$ and $k_i \in \Z$.
                    since $c$ is a root of the polynomial, by definition,
                    \[
                    p(c) = a_ic^i + a_{i-1}c^{i-1} \ldots a_2c^2 + a_1c +a_0 = 0
                    \]
                    where $p_1(c)$ can be written as 
                    \[
                    p_1(c) = \mathbf{d}p_0(c) = \mathbf{d} \cdot 0 = 0
                    \]
                    and $c$ is the root of the polynomial $p_1(x)$ where all coefficients are integers.

                    
                \end{proof}
                \newpage

                \item[33.] When expressions of the form $(x-r)(x-s)$ are multiplied out, a quadratic polynomial is obtained. 
                    \begin{enumerate}[a.]
                        \item what can be said about the coefficients of the polynomials obtained when $r,s$ are odd integers, and even integers. and what is one is even and the other is odd. \\
                        let both $r,s$ be even integers, therefore for some $k_1, k_2 \in \Z$, $r = 2k_1$ and $s = 2k_2$. The polynomial can be obtained by
                        \[
                        (x-2k_1)(x-2k_2) = x^2 - 2k_2x - 2k_1x + 4k_1k_2 = x^2 - 2(k_1+k_2)x + 4k_1k_2
                        \]
                        define $k_3 = k_1 + k_2 \in \Z$ and $k_4 = 2k_1k_2 \in \Z$, since integers are closed under addition and multiplication, we can rewrite the polynomial as
                        \[
                        x^2 - 2k_3x + 2k_4
                        \]
                        where $2k_3 \in 2\Z$ and $2k_4 \in \Z$. Therefore polynomial obtained this way will have even coefficients in front of the first degree term and as the constant. 
                        \\\\
                        Now, let both $r,s$ be odd integers, therefore for some $k_1, k_2 \in \Z$, $r = 2k_1+1$ and $s = 2k_2+1$. The polynomial can be obtained by
                        \[
                        \sqbkt{x-(2k_1+1)}\sqbkt{x-(2k_2+1)} = x^2 - (2k_2+1)x - (2k_1+1)x + (2k_1+1)(2k_2+1) 
                        \]
                        \[
                        = x^2 - 2(k_1+k_2 + 1)x + 4k_1k_2+2k_1k+2k_2 + 
                        \]
                        \[
                        = x^2 - 2k_3x + (2k_4+1)
                        \]
                        where $k_3 = k_1+k_2 + 1 \in \Z$ and $ k_4 = 2k_1k_2+k_1k+k_2 \in \Z$ since integers are closed under multiplication and addition, and $2k_3 \in 2\Z$ and $ 2k_4 + 1 \in \Z - 2\Z$ therefore the obtained polynomial has even coefficient for the first degree term and odd number as its constant. 
                        \\\\
                        Let \( r = 2k_1 \) and \( s = 2k_2 + 1 \) for some \( k_1, k_2 \in \mathbb{Z} \). The polynomial can be obtained by:
                        \[
                        (x - 2k_1)(x - (2k_2 + 1)) = x^2 - (2k_2 + 1)x - 2k_1 x + 2k_1 (2k_2 + 1)
                        \]
                        \[
                        = x^2 - (2k_2 + 1 + 2k_1)x + (4k_1 k_2 + 2k_1)
                        \]
                        Define \( k_3 = k_1 + k_2 \in \mathbb{Z} \) and \( k_4 = 2k_1 k_2 + k_1 \in \mathbb{Z} \). Since integers are closed under addition and multiplication, we can rewrite the polynomial as:
                        \[
                        x^2 - (2k_3 + 1)x + 2k_4
                        \]
                        where \( 2k_3 + 1 \in \mathbb{Z} - 2\mathbb{Z} \) and \( 2k_4 \in 2\mathbb{Z} \). Therefore, the polynomial obtained this way will have an odd coefficient for the first degree term and an even constant.
                    \item The provided polynomial $x^2 - 1253x + 255$ cannot be written in the form $(x-r)(x-s)$ where $r,s$ are integers since it has all odd coefficients and constant, which does not match our observations for any cases above. 
                       
                        
                    \end{enumerate}
                 \newpage

                 \item[36.] The questions asks to prove the following argument
                 \[
                 \forall r,s \in \Q \paren{r+s \in \Q}
                 \]
                 And the proof provided did not show that for every single rational number, there sum is a rational number. It only proved that \[
                 \exists r,s \in \Q \paren{r+s \in \Q}
                 \] 
                 there exists rational number such that there exists another rational number where their sum is rational. 
    
            \end{enumerate}


            \newpage
            
    \item Section 4.4 \#28, 29, 30, 37
        \begin{enumerate}
            \item[28.] $\forall a,b,c \in \Z \paren{ a \mid bc \rightarrow a\mid b \vee a \mid c}$
            \begin{proof}[disproof]
                let $a = 6 \in \Z$, let $b = 2 \in \Z$ and $ c = 3 \in \in \Z$, then
                \[
                a \nmid b, \quad a \nmid c, \quad a\mid bc
                \]
            \end{proof}

            \item[29.] \( \forall a,b \in \Z \paren{ a \mid b \then a^2 \mid b^2}\)
            \begin{proof}[disproof]
                let $a=3 \in \Z$ and $b = 6 \in \Z$, then $a^2 = 9$ and $b^2 = 36$
                \[
                a \mid b,  \quad \text{but }a^2 \nmid b^2 
                \]
            \end{proof}

            \item[30.] \( \forall a,n \in \Z \paren{ a \mid n^2 \wedge a \leq n \then a \mid n}\)
            \begin{proof}[disproof]
                let $ a = 4 \in \Z$ and $ n = 6 \in \Z$, $n^2 = 36 \in \Z$, then
                \[
                a\leq n, \;a \mid  n^2 \quad \text{but } a \nmid n
                \]
                
            \end{proof}
        \item[37.] Factor the following integers into prime factors
            \begin{enumerate}[a.]
                \item $1176 = 2^3 \cdot 3 \cdot 7^2$
                \item $5733 = 3^2 \cdot 7^2 \cdot 13$
                \item $3675 = 3 \cdot 5^2 \cdot 7^2$
            \end{enumerate}
        \end{enumerate}

    \newpage
    
    \item Section 4.4 \#45, 48
        \begin{enumerate}
            \item[45.] Prove that if $n$ is any non-negative integer whose decimal representation ends in 5, then $5 \mid n$
                \begin{proof}
                    let $n$ be an integer whose decimal representation ends in $5$ , and can be written as follows
                    \[
                    n = 5 + \sum_{i=1}^{k} k_i \cdot 10^k
                    \]
                    where $k \in \Z^+ \cup \zero$ and $k_i$ is any integer chosen from $\{0,1,2, \ldots 8, 9\}$, now we multiply both side by $\frac{1}{5}$, we have
                    \[
                    \frac{n}{5} = \frac{1}{5} \paren{5+ \sum_{i=1}^{k} k_i \cdot 10^k} 
                    = \frac{2}{10} \paren{5 + \sum_{i=1}^{k} k_i \cdot 10^k}
                    \]
                    through algebra, we obtain
                    \[
                    \frac{n}{5} = 1 + \sum_{i=1}^{k} \frac{2k_i \cdot 10^k}{10} = 1+ \sum_{i=1}^{k} 2k_i \cdot 10^{k-1}
                    \]
                    We see that $ \frac{n}{5} \in \Z$ since integer is closed under addition and multiplication, and $ 5 \leq n$, then $5 \mid n$ if $n$ has a decimal representation that ends in 5.
                    
                \end{proof}

            \item[48.] Prove that for any non-negative integer $n$, if the sum of the digits of $n$ is divisible by 3, then $n$ is divisible by 3. 
                \begin{proof}
                    let $n$ be an integer, whose decimal representation can be written as follows
                    \[
                    n = \overline{d_id_{i-1}\ldots d_1d_0}
                    \]
                    Where $ i \in \Z^+ \cup \zero$ and It can be rewritten in the form
                    \[
                    n = d_i\cdot \paren{ \paren{10^i -1} +1}  + d_{i-1}\cdot \paren{\paren{10^{i-1} -1} +1}
                    \ldots d_1 (10 -1 +1)+d_0
                    \]
                    Multiply some of the elements out of the parenthesis, we get, 
                    \[
                    n = d_i + \paren{10^i - 1} + d_{i-1}+ \paren{10^{i-1}-1} \ldots d_1 + 9 + d_0
                    \]
                    notice that $\paren{10^i - 1}$ can be rewritten as 
                    \[
                    \paren{10^i - 1} = \sum_{k=0}^{i-1} 9 \cdot 10^k
                    \]
                    which is divisible by 3, so if $d_i + d_{i-1} \ldots d_1 + d_0$ is divisible by 3, then the number $n$ is divisible by 3. 
                \end{proof}
        \end{enumerate}

        \newpage

        
    \item Section 4.5 \#17,21

        \begin{enumerate}
            \item[17.] Prove directly from the definitions using division into two cases that 
            \[
            \forall n \in \Z \paren{n^2 - n +3 \in \odd }
            \]
            \begin{proof}:\\
                Case 1: Let \( n \in \odd \). Then, for some integer \( k_1 \),
                    \[ n = 2k_1 + 1. \]
                    
                    Substitute \( n = 2k_1 + 1 \) into \( n^2 - n + 3 \):
                    \[
                    n^2 - n + 3 = (2k_1 + 1)^2 - (2k_1 + 1) + 3.
                    \]
                    and
                    \[
                    n^2 - n + 3 = 4k_1^2 + 4k_1 + 1 - (2k_1 + 1) + 3.
                    \]
                    Simplify the expression:
                    \[
                    4k_1^2 + 4k_1 + 1 - 2k_1 - 1 + 3 = 4k_1^2 + 2k_1 + 3.
                    \]
                    Factor out the 2 from the first two terms:
                    \[
                    4k_1^2 + 2k_1 + 3 = 2z_1 + 1.
                    \]
                    Where $z_1 = 2k_1^2 + k_1+1 \in \Z$, since integers are closed under addition and multiplication, and $2z_1+1 \in \odd$,  Therefore,
                    \[
                    n^2 - n + 3 \text{ is odd when } n \text{ is odd}.
                    \]
                    
                    \textbf{Case 2:} \( n \) is even.
                    
                    Let \( n \in \mathbb{Z} \) be an even integer. Then, for some integer \( k_2 \),
                    \[ n = 2k_2. \]
                    
                    Substitute \( n = 2k_2 \) into \( n^2 - n + 3 \):
                    \[
                    n^2 - n + 3 = (2k_2)^2 - 2k_2 + 3.
                    \]
                    and
                    \[
                    n^2 - n + 3 = 4k_2^2 - 2k_2 + 3.
                    \]
                    Factor out the 2 from the first two terms:
                    \[
                    4k_2^2 - 2k_2 + 3 = 2z_2 + 1.
                    \]
                    Where $z_2 = 2k_2^2 - k_2 +1 \in \Z$, since integers are closed under addition and multiplication, and $2z_2 +1  \in \odd$,  Therefore
                    \[
                    n^2 - n + 3 \text{ is odd when } n \text{ is even}.
                    \]
                    
                    Since \( n^2 - n + 3 \) is odd in both cases (whether \( n \) is odd or even), we have proven that:
                    \[
                    \forall n \in \mathbb{Z}, \, n^2 - n + 3 \text{ is odd}.
                    \]
            \end{proof}

            \item[21.] Suppose $b$ is any integer. If $b \bmod 12 = 5$, what is $8b \bmod 12$? 
                \begin{proof}
                    Let $b \in \Z$, and suppose $ b \bmod 12 = 5$, by quotient remainder theorem, we have
                    \[
                    b = 12q+5
                    \]
                    Where $q \in \Z$, and therefore
                    \[
                    8b = 8(12q+5) = 12(8q)+40 = 12(8q)+3\times 12 +4 = 12(8q+3)+4
                    \]
                    Therefore $8b \bmod 12 = 4$
                \end{proof}
        \end{enumerate}

        \newpage

    \item Section 4.5 \#25, 31(a), 33
        \begin{enumerate}
            \item[25.] Prove that for all integers $a,b$, if $a \bmod 7 = 5$ and $b \bmod 7 = 6$ then $ab \bmod 7 = 2$. 
                \begin{proof} Let $a,b \in \Z $, suppose $a \bmod 7 = 5$ and $b \bmod 7 = 6$, and by quotient remainder theorem, we can rewrite $a,b$ as the follows for some $q \in \Z$
                \[
                a = 7q+5 \quad \text{and} b = 7q + 6
                \]
                by substitution, we can find that
                \[
                ab = (7q+5)(7q+6) = 7\paren{7q^2}+7\paren{6q}+7\paren{5q}+30
                \]
                Simplify the expression
                \[
                ab = 7\paren{7q^2}+7\paren{11q}+ (4\times 7 + 2)
                \]
                And through factoring, we can rewrite it in the following form
                \[
                ab = 7 \paren{7q^2 + 11q + 4}+2
                \]
                let $q_0 = 7q^2 + 11q + 4 \in \Z$ since integer is closed under multiplication and addition, and we can rewrite $ab$ as the form $ab = 7q_0 + 2$, therefore
                \[
                ab \bmod 7 = 2
                \]
                    
                \end{proof}

            \item[31(a).] Prove that for all integers $m,n$, $m+n$ and $m-n$ are either both odd or both even. 
                \begin{proof}
                Let \(m, n \in \mathbb{Z}\). We will consider three cases: when both \(m\) and \(n\) are even, when both \(m\) and \(n\) are odd, and when one is even and the other is odd.
                
                \textbf{Case 1: Both \(m\) and \(n\) are even.}
                
                If \(m, n \in 2\mathbb{Z}\), then for some \(k_1, k_2 \in \mathbb{Z}\),
                \[ m = 2k_1 \quad \text{and} \quad n = 2k_2. \]
                
                Then,
                \[ m+n = 2k_1 + 2k_2 = 2(k_1+k_2), \]
                which is even, and
                \[ m-n = 2k_1 - 2k_2 = 2(k_1-k_2), \]
                which is also even. As \((k_1+k_2) \in \mathbb{Z}\) and \((k_1-k_2) \in \mathbb{Z}\) since integers are closed under addition and subtraction, both \(m+n\) and \(m-n\) are even.
                
                \textbf{Case 2: Both \(m\) and \(n\) are odd.}
                
                If \(m, n \in \mathbb{Z}\) are odd, then for some \(k_3, k_4 \in \mathbb{Z}\),
                \[ m = 2k_3 + 1 \quad \text{and} \quad n = 2k_4 + 1. \]
                
                Then,
                \[ m+n = (2k_3 + 1) + (2k_4 + 1) = 2k_3 + 2k_4 + 2 = 2(k_3 + k_4 + 1), \]
                which is even, and
                \[ m-n = (2k_3 + 1) - (2k_4 + 1) = 2k_3 - 2k_4 = 2(k_3 - k_4), \]
                which is also even. As \((k_3 + k_4 + 1) \in \mathbb{Z}\) and \((k_3 - k_4) \in \mathbb{Z}\) since integers are closed under addition and subtraction, both \(m+n\) and \(m-n\) are even.
                
                \textbf{Case 3: One of \(m\) or \(n\) is even, and the other is odd.}
                
                Assume \(m\) is even and \(n\) is odd without loss of generality. Then, for some \(k_5, k_6 \in \mathbb{Z}\),
                \[ m = 2k_5 \quad \text{and} \quad n = 2k_6 + 1. \]
                
                Then,
                \[ m+n = 2k_5 + (2k_6 + 1) = 2k_5 + 2k_6 + 1 = 2(k_5 + k_6) + 1, \]
                which is odd, and
                \[ m-n = 2k_5 - (2k_6 + 1) = 2k_5 - 2k_6 - 1 = 2(k_5 - k_6) - 1, \]
                which is also odd. As \((k_5 + k_6) \in \mathbb{Z}\) and \((k_5 - k_6) \in \mathbb{Z}\) since integers are closed under addition and subtraction, both \(m+n\) and \(m-n\) are odd.
                
                Thus, for all integers \(m\) and \(n\), \(m+n\) and \(m-n\) are either both odd or both even.
                \end{proof}
                
            \item[33.]Given any integers, $a,b,c$, if $a-b$ is odd and $b-c$ is even, what can you say about $a-c$?
                \begin{proof}
                Let \(a, b, c \in \mathbb{Z}\). We know that \(a-b\) is odd and \(b-c\) is even.
                
                Since \(a-b\) is odd, there exist integers \(k_1\) such that
                \[ a - b = 2k_1 + 1. \]
                
                Since \(b-c\) is even, there exist integers \(k_2\) such that
                \[ b - c = 2k_2. \]
                
                Now consider \(a-c\):
                \[
                a - c = (a - b) + (b - c).
                \]
                
                Substitute the expressions for \(a - b\) and \(b - c\):
                \[
                a - c = (2k_1 + 1) + 2k_2 = 2k_1 + 2k_2 + 1 = 2(k_1 + k_2) + 1.
                \]
                
                Since \(k_1 + k_2 \in \mathbb{Z}\), we can see that \(a - c = 2(k_1 + k_2) + 1 \in \odd\),
                
                Therefore, if \(a-b\) is odd and \(b-c\) is even, \(a-c\) must be odd.
                \end{proof}
                
        \end{enumerate}

        \newpage


    \item Section 4.5 \#38, 42, 47
        \begin{enumerate}
            \item[38.] For every integer $m$, $m^2 = 5k$, or $m^2 = 5k+1$, or $m^2 = 5k+4$ for some integer k. 
                \begin{proof}
                Let \( m \in \mathbb{Z} \). We will consider the possible remainders when \( m \) is divided by 5. 
                
                \textbf{Case 1:} Suppose \( m \bmod 5 = 0 \). Then by the quotient remainder theorem, for some \( q \in \mathbb{Z} \), \( m \) can be rewritten as
                \[ m = 5q. \]
                If we square it, we obtain
                \[ m^2 = (5q)^2 = 25q^2 = 5(5q^2). \]
                Let \( k_0 = 5q^2 \in \mathbb{Z} \). Since integers are closed under multiplication, it can be seen that this is in the form \( m^2 = 5k_0 \).
                
                \textbf{Case 2:} Suppose \( m \bmod 5 = 1 \). Then we can rewrite \( m \) as follows
                \[ m = 5q + 1. \]
                Consequently,
                \[ m^2 = (5q + 1)^2 = 25q^2 + 10q + 1 = 5(5q^2 + 2q) + 1. \]
                Let \( k_1 = 5q^2 + 2q \in \mathbb{Z} \). Since integers are closed under multiplication and addition, we see that \( m^2 \) is in the form \( m^2 = 5k_1 + 1 \).
                
                \textbf{Case 3:} Suppose \( m \bmod 5 = 2 \). Then we can rewrite \( m \) as follows
                \[ m = 5q + 2. \]
                Consequently,
                \[ m^2 = (5q + 2)^2 = 25q^2 + 20q + 4 = 5(5q^2 + 4q) + 4. \]
                Let \( k_2 = 5q^2 + 4q \in \mathbb{Z} \). Since integers are closed under multiplication and addition, we see that \( m^2 \) is in the form \( m^2 = 5k_2 + 4 \).
                
                \textbf{Case 4:} Suppose \( m \bmod 5 = 3 \). Then we can rewrite \( m \) as follows
                \[ m = 5q + 3. \]
                Consequently,
                \[ m^2 = (5q + 3)^2 = 25q^2 + 30q + 9 = 5(5q^2 + 6q + 1) + 4. \]
                Let \( k_3 = 5q^2 + 6q + 1 \in \mathbb{Z} \). Since integers are closed under multiplication and addition, we see that \( m^2 \) is in the form \( m^2 = 5k_3 + 4 \).
                
                \textbf{Case 5:} Suppose \( m \bmod 5 = 4 \). Then we can rewrite \( m \) as follows
                \[ m = 5q + 4. \]
                Consequently,
                \[ m^2 = (5q + 4)^2 = 25q^2 + 40q + 16 = 5(5q^2 + 8q + 3) + 1. \]
                Let \( k_4 = 5q^2 + 8q + 3 \in \mathbb{Z} \). Since integers are closed under multiplication and addition, we see that \( m^2 \) is in the form \( m^2 = 5k_4 + 1 \).
                
                In all cases, we have shown that \( m^2 \) is either of the form \( 5k \), \( 5k+1 \), or \( 5k+4 \) for some integer \( k \).
                \end{proof}

             \item[42.] Prove that
                \[
                \forall r \in \R \; \forall c \in \R^+ \cup \zero \paren{ -c \leq r \leq c \leftrightarrow \abso{r}\leq c}
                \]
                \begin{proof}
                Let \( r \in \mathbb{R} \) and \( c \in \mathbb{R}^+ \cup \{0\} \).
                
                First, we prove the forward direction:
                \[ -c \leq r \leq c \rightarrow |r| \leq c. \]
                
                Assume \( -c \leq r \leq c \). By the definition of absolute value, \( |r| \) is given by:
                \[ 
                |r| = 
                \begin{cases} 
                r & \text{if } r \geq 0 \\
                -r & \text{if } r < 0 
                \end{cases}.
                \]
                
                Since \( -c \leq r \leq c \):\\
                - If \( r \geq 0 \), then \( 0 \leq r \leq c \), so \( |r| = r \leq c \).\\
                - If \( r < 0 \), then \( -c \leq r < 0 \), so \( -c \leq -r \leq 0 \), giving \( 0 \leq -r \leq c \),\\ thus \( |r| = -r \leq c \).
                
                In both cases, we have \( |r| \leq c \).
                
                Now, we prove the backward direction:
                \[ |r| \leq c \rightarrow -c \leq r \leq c. \]
                
                Assume \( |r| \leq c \). By the definition of absolute value, \( |r| \leq c \) means:
                \[ 
                |r| = 
                \begin{cases} 
                r & \text{if } r \geq 0 \\
                -r & \text{if } r < 0 
                \end{cases}
                \leq c.
                \]
                
                - If \( r \geq 0 \), then \( |r| = r \leq c \), thus \( 0 \leq r \leq c \). This implies \( 0 \leq r \leq c \).\\
                - If \( r < 0 \), then \( |r| = -r \leq c \), thus \( -r \leq c \) which implies \( -c \leq r < 0 \).\\
                
                In both cases, we have \( -c \leq r \leq c \).
                
                Therefore, we have shown that:
                \[
                \forall r \in \mathbb{R} \; \forall c \in \mathbb{R}^+ \cup \{0\} \left( -c \leq r \leq c \leftrightarrow |r| \leq c \right).
                \]
                
                \end{proof}

                \newpage

            \item[47.] If $m,n$ and $d$ are integers, and $d>0 $, $ d\mid (m-n)$, what is the relationship between $m \bmod d$ and $n \bmod d$, prove your answer. 
                \begin{proof}
                Let $m, n \in \mathbb{Z}$ and $d \in \mathbb{Z}^+$. Rewrite $m-n$ as
                \[
                m - n = dq + r
                \]
                for some integer $q$ and non-negative integer $r$, where $0 \leq r < d$. Since $d \mid (m-n)$, by the quotient remainder theorem, we can rewrite $m-n$ as follows
                \[
                m-n = dq
                \]
                Now, express $m$ and $n$ in terms of $d$:
                \[
                m = dq_m + r_m
                \]
                \[
                n = dq_n + r_n
                \]
                for some integers $q_m, q_n$ and non-negative integers $r_m, r_n$ where $0 \leq r_m < d$ and $0 \leq r_n < d$.
                
                Substitute these into the expression $m - n$:
                \[
                m - n = (dq_m + r_m) - (dq_n + r_n) = d(q_m - q_n) + (r_m - r_n).
                \]
                notice that $q = q_m - q_n$, and $\paren{r_m - r_n} = 0$, therefore we can make the following conclusion
                \[
                r_m = r_n
                \]
                and $m \bmod d$ and $n \bmod d$ would have the same remainder. 
                

                \end{proof}


                    
        \end{enumerate}
\newpage
    \item Use contradiction to prove the following statement
        \[
        \forall x \in \mathbb{R} \left( \left( \forall \epsilon > 0 : |x| < \epsilon \right) \rightarrow x = 0 \right)
        \]
        
        \begin{proof}
        Assume the negation of the statement:
        \[
        \exists x \in \mathbb{R} \left( \left( \forall \epsilon > 0 : |x| < \epsilon \right) \wedge x \neq 0 \right)
        \]
        
        This means that there exists some real number \( x \) such that for all positive numbers \( \epsilon \), \( |x| < \epsilon \) and \( x \neq 0 \).
        
        By the definition of absolute value, \( |x| \geq 0 \) and \( |x| \neq 0 \) since \( x \neq 0 \). Therefore, \( |x| > 0 \).
        
        Now, since \( |x| > 0 \), we can choose \( \epsilon = |x| \). This choice of \( \epsilon \) implies:
        \[
        |x| < \epsilon = |x|.
        \]
        
        This is a contradiction because \( |x| \) cannot be strictly less than \( |x| \). Therefore, our assumption that there exists \( x \in \mathbb{R} \) and \( \epsilon > 0 \) such that \( |x| < \epsilon \) and \( x \neq 0 \) must be false.
        
        Hence, we conclude:
        \[
        \forall x \in \mathbb{R} \left( \left( \forall \epsilon > 0 : |x| < \epsilon \right) \rightarrow x = 0 \right).
        \]
        \end{proof}
    \newpage
    \item Section 4.7 \#9, 18
        \begin{enumerate}
            \item[9.] Complete the 2 parts
                \begin{enumerate}[a.]
                    \item Part a: When asked to prove that the difference of any irrational number and any rational number is irrational, a student began, “Suppose not. That is, suppose the difference of any irrational number and any rational number is rational.” What is wrong with beginning the proof in this way? \\
                        The catch here is, the student did not assume the negation of the original statement. The original statement says that
                        \[
                        \forall r \in \Q \; \forall s \in \R - \Q \paren{r-s\in \R - \Q}
                        \]
                        The correct negation to assume for prove via contradiction is
                        \[
                        \exists r \in \Q \; \exists s \in \R - \Q \paren{r-s \notin \R - \Q}
                        \]
                        However, the student's statement was
                        \[
                        \forall r \in \Q \; \forall s \in \R - \Q \paren{r-s\notin \R - \Q}
                        \]
                        which is not the logical negation of the original statement. 
                    \item Prove that the difference of any irrational number and any rational number is irrational.
                        \begin{proof}[proof]
                            choose $r\in \Q$ and $s \in \R-\Q$. Suppose that $r-s \notin \Q$, then for some $m_1, m_2 \in \Z$ and $n_1, n_2 \in \Z - \zero$.
                            \[
                            r = \frac{m_1}{n_1} \text{ and } r-s = 
                            \]
                            and by algebra and substitution
                            \[
                            -s = r-s + r = \frac{m_2}{n_2} + \frac{m_1}{n_1}
                            \]
                            \[
                            -s = \frac{m_1n_2 + m_2n_1}{n_1n_2}
                            \]
                            notice that $m_1n_2 + m_2n_1 \in \Z$ since integer is closed under multiplication and addition, and $n_1n_2 \in \Z - \zero$, we can conclude that $s \in \Q$, which contradicts with the presumption that $s \in \R - \Q$. Therefore the original statement is true via contradiction. 
                        
                        
                        \end{proof}
                            

                \end{enumerate}

            \newpage
            \item[18.] If $a,b \in \Q$, $b \neq 0$, and $r \in \R - \Q$, then $a+br \in \R - \Q$.
                \[
                \forall a \in \Q \; \forall b \in \Q - \zero \; \forall r \in \R - \Q \paren{
                a+br \in \R - \Q
                }
                \]
                \begin{proof}
                    suppose for some $a \in \Q$, for some $b \in \Q - \zero$, and for some $ r\in \R - \Q$, that $a+br \in \Q$, then by definition, for some $m_1,m_3 \in \Z$, $n_1,n_2,n_3,m_2 \in \Z - \zero$
                    \[
                    a = \frac{m_1}{n_1} \quad b = \frac{m_2}{n_2} \quad a+br = \frac{m_3}{n_3}
                    \]
                    by algebra, we can express $r$ as
                    \[
                    r = \frac{1}{b}\paren{\frac{m_3}{n_3} - a}
                    \]
                    by substitution, we have
                    \[
                    r = \frac{n_2}{m_2}\paren{\frac{m_3}{n_3} - \frac{m_1}{n_1}}
                    \]
                    by algebra, we have
                    \[
                    r = \frac{n_2}{m_2}\cdot \frac{m_3n_1 - m_1n_3}{n_1n_3} = \frac{m_3n_1n_2 - m_1n_3n_2}{n_1n_2n_3}
                    \]
                    We see that $m_3n_1n_2 - m_1n_3n_2 \in \Z$ and $n_1n_2n_3 \in \Z - \zero$, therefore $r \in \Q$, which contradicts with the assumption $r \in \R - \Q$, therefore we proved the statement by contradiction. 
                \end{proof}
        \end{enumerate}
    \newpage
    \item Section 4.7 \#22, 24
        \begin{enumerate}
            \item[22.] Consider the statement \textit{For every real number $r$, if $r^2$ is irrational then $r$ is irrational.}
                \begin{enumerate}[a.]
                    \item Write what you would suppose and what you would need to show to prove this statement by contradiction.
                    First, we would write the statement as follows
                    \[
                    \forall r \in \R \paren{r^2 \in \R - \Q \then r \in \R - \Q}
                    \]
                    Then, write the negation of the statement
                    \[
                    \exists r \in \R \paren{r^2 \in \R - \Q \wedge r \notin \R - \Q}
                    \]
                    We would assume the above negation is true, and show that this assumption would lead to a contradiction, thus proving the negation is false therefore the original statement is true.
                    \item Contraposition\\
                    To prove it with contrapositive, we would first rewrite the statement in its contrapositive form. 
                    \[
                    \forall r \in \R \paren{ r \notin \R - \Q \then r^2 \notin \R - \Q}
                    \]
                    And then, we let $r \in \R$ and assume $r \notin \R - \Q$, then we would prove that $r^2 \notin \R - \Q$. And since contrapositive is logically equivalent to the original statement, this is a direct prove of the original statement. 
                \end{enumerate}

            \item[24.] The reciprocal of any irrational number is irrational.

                \begin{enumerate}[I.]
                    \item Prove by contradiction. First, we would obtain the negation of
                    \[
                    \forall r \in \mathbb{R} - \mathbb{Q} \left( \frac{1}{r} \in \mathbb{R} - \mathbb{Q} \right)
                    \]
                    its negation is
                    \[
                    \exists r \in \mathbb{R} - \mathbb{Q} \left( \frac{1}{r} \in \mathbb{Q} \right)
                    \]
                    \begin{proof}
                        Assume for some \( r \in \mathbb{R} - \mathbb{Q} \), \( \frac{1}{r} \in \mathbb{Q} \), then for some \( m, n \in \mathbb{Z} - \{0\} \) (0 is not an irrational number), we can write \( \frac{1}{r} \) as follows
                        \[
                        \frac{1}{r} = \frac{m}{n}
                        \]
                        and therefore we can write \( r \) as
                        \[
                        r = \frac{n}{m}
                        \]
                        Notice \( m, n \in \mathbb{Z} \), therefore \( r \) is a rational number. This contradicts the assumption that \( r \) is an irrational number. Therefore, the statement is true via contradiction.
                    \end{proof}
                    
                    \item Prove by contraposition. We would first rewrite the original statement.
                    \[
                    \forall r \in \mathbb{R} \left( r \in \mathbb{R} - \mathbb{Q} \rightarrow \frac{1}{r} \in \mathbb{R} - \mathbb{Q} \right)
                    \]
                    Its contrapositive is
                    \[
                    \forall r \in \mathbb{R} \left( \frac{1}{r} \notin \mathbb{R} - \mathbb{Q} \rightarrow r \notin \mathbb{R} - \mathbb{Q} \right)
                    \]
                    \begin{proof}
                        To prove the contrapositive, assume \( \frac{1}{r} \notin \mathbb{R} - \mathbb{Q} \). This means \( \frac{1}{r} \in \mathbb{Q} \). Hence, for some \( m, n \in \mathbb{Z} - \{0\} \),
                        \[
                        \frac{1}{r} = \frac{m}{n}
                        \]
                        This implies
                        \[
                        r = \frac{n}{m}
                        \]
                        Since \( m \) and \( n \) are integers, \( r \) is a rational number. Therefore, \( r \notin \mathbb{R} - \mathbb{Q} \).
                
                        Thus, we have shown that
                        \[
                        \frac{1}{r} \notin \mathbb{R} - \mathbb{Q} \rightarrow r \notin \mathbb{R} - \mathbb{Q}
                        \]
                
                        Since we have proven the contrapositive, the original statement is true:
                        \[
                        \forall r \in \mathbb{R} \left( r \in \mathbb{R} - \mathbb{Q} \rightarrow \frac{1}{r} \in \mathbb{R} - \mathbb{Q} \right)
                        \]
                    \end{proof}
                \end{enumerate}
                
        \end{enumerate}

    \newpage
    \item Section 4.7 \#28. Prove using contradiction and contrapositive.
        \[
        \forall a,b,c \in \Z \paren{ a\mid b \wedge a \nmid c \then a \nmid (b+c)}
        \]
        The negation of the given statement is
        \[
        \exists a,b,c \in \Z \paren{ a \mid b \wedge a \nmid c \wedge a \mid (b+c)}
        \]
        
        First we would prove via contradiction.
        \begin{proof}[proof 4.7.28-I]
            For some $a,b,c \in \Z$, assume $a \mid b \wedge a \nmid c \wedge a \mid (b+c)$. By quotient remainder theorem, we can write that for some $q_1,q_2 \in \Z$ and $r_1,r_2 \in \Z$
            \[
            b = aq_1 + r_1 \quad (b+c) = aq_2 + r_2
            \]
            where $0 \leq r_1 < a$ and $ 0 \leq r_2 < a$ by algebra and substitution, we obtain
            \[
            c = b+c - b = aq_2 + r_2 - aq_1 + r_1
            \]
            through more algebra, we have
            \[
            c = a \paren{q_1 - q_2} + \paren{r_1 - r_2 }
            \]
            since $a \mid b \wedge a \mid (b+c)$, therefore $r_1 = r_2 = 0$, we have
            \[
             c = a \paren{q_1 - q_2}
            \]
            this implies that $a \mid  c$, which contradicts the assumption that $a \nmid c$, therefore we proved the statement via contradiction. 
        \end{proof}
            The contrapositive of the statement is
            \[
            \forall a,b,c \in \mathbb{Z} \left( a \mid (b+c) \rightarrow a \nmid b \vee a \mid c \right).
            \]
            \begin{proof}[Proof 4.7.28-II]
            Let \( a, b, c \in \mathbb{Z} \). Assume \( a \mid (b+c) \). By the quotient remainder theorem, we can rewrite it as follows for some \( q_0 \in \mathbb{Z} \) and \( r_0 \in \mathbb{Z} \) where \( 0 \leq r_0 < a \):
            \[
            b + c = aq_0 + r_0.
            \]
            Since \( a \mid (b + c) \), it follows that \( r_0 = 0 \). Hence,
            \[
            b + c = aq_0.
            \]
            
            Now consider two cases: \( a \mid b \) and \( a \nmid b \).
            
            \textbf{Case 1: \( a \mid b \)}
            
            If \( a \mid b \), then \( b = aq_1 \) for some \( q_1 \in \mathbb{Z} \). Substituting \( b = aq_1 \) into the equation \( b + c = aq_0 \), we get
            \[
            aq_1 + c = aq_0.
            \]
            Solving for \( c \), we get
            \[
            c = aq_0 - aq_1 = a(q_0 - q_1).
            \]
            This implies \( a \mid c \).
            
            \textbf{Case 2: \( a \nmid b \)}
            
            If \( a \nmid b \), then \( a \nmid b \) holds by assumption.
            
            Combining these cases, we have shown that if \( a \mid (b+c) \), then either \( a \nmid b \) or \( a \mid c \).
            
            Therefore, we have proved the contrapositive, which implies the original statement:
            \[
            \forall a,b,c \in \mathbb{Z} \left( a \mid b \wedge a \nmid c \rightarrow a \nmid (b+c) \right).
            \]
            \end{proof}

        \newpage

    \item Section 4.7 \#31
        \begin{enumerate}[a.]
            \item Prove by contraposition that
                \[
                \forall n,r,s \in \Z^+ \paren{rs \leq n \then r \leq \sqrt{n} \vee s \leq \sqrt{n}}
                \]
                The contraposition of the statement is
                \[
                \forall n,r,s \in \Z^+ \paren{ r > \sqrt{n} \wedge s > \sqrt{n} \then rs > n}
                \]
                    \begin{proof}
                        let $n,r,s \in \Z^+$, suppose that $r > \sqrt{n}$ and $ s > \sqrt{n}$. Since $n,r,s >0$, therefore 
                        \[
                        rs > 0 \quad \sqrt{n}>0
                        \]
                        since
                        \[
                        r>\sqrt{n}>0 \quad s > \sqrt{n} >0
                        \]
                        therefore
                        \[
                        rs > n > 0
                        \]
                        which proves the contraposition of the statement, and that implies the original statement is true.
                    \end{proof}
            \item Prove that for each integer \( n > 1 \), if \( n \) is not prime then there exists a prime number \( p \) such that \( p \leq \sqrt{n} \) and \( p \mid n \).
            Let \( \mathbb{P} \) denote the set of prime numbers.
            \[
            \forall n \in \mathbb{Z}^+ - \{1\} \left( n \notin \mathbb{P} \rightarrow \left( \exists p \in \mathbb{P} : p \leq \sqrt{n} \wedge p \mid n \right) \right)
            \]
            The negation of the statement is
            \[
            \exists n \in \mathbb{Z}^+ - \{1\} \left( n \notin \mathbb{P} \wedge \left( \forall p \in \mathbb{P} : p > \sqrt{n} \vee p \nmid n \right) \right)
            \]
            
            \begin{proof}[Proof by contradiction]
            Choose \( n \in \mathbb{Z}^+ \) with \( n > 1 \). Suppose that \( n \) is not a prime and for all prime numbers \( p \), \( p > \sqrt{n} \) or \( p \nmid n \). Since \( n \) is not a prime number, then by the fundamental theorem of arithmetic, \( n \) can be written as
            \[
            n = \prod_{i=1}^{k} p_i^{\alpha_i}
            \]
            for some \( k \in \mathbb{Z}^+ \) and prime numbers \( p_i \) and positive integers \( \alpha_i \). The square root of \( n \) can be written as
            \[
            \sqrt{n} = \prod_{i=1}^{k} p_i^{\frac{\alpha_i}{2}}.
            \]
            
            Since \( n \) is composite number and $n>1$, there exists at least one prime factor \( p_i \) such that \( p_i \leq \sqrt{n} \). Let's denote this prime factor by \( p \).
            
            By our assumption, for all prime numbers \( p \), either \( p > \sqrt{n} \) or \( p \nmid n \). However, we have already identified a prime factor \( p \leq \sqrt{n} \) that divides \( n \), which contradicts our assumption.
            
            Therefore, our initial assumption that \( n \) is composite and that for all prime numbers \( p \), \( p > \sqrt{n} \) or \( p \nmid n \) must be false. This contradiction shows that if \( n \) is not prime, there must exist a prime number \( p \) such that \( p \leq \sqrt{n} \) and \( p \mid n \).
            
            Hence, we have proved the statement by contradiction.
            \end{proof}
        \item The contrapositive of the statement is
            \[
            \forall n \in \mathbb{Z}^+ - \{1\} \left( \left( \forall p \in \mathbb{P} : p > \sqrt{n} \vee p \nmid n \right) \rightarrow n \in \mathbb{P} \right)
            \]
            Since this statement is logically equivalent to the original, it is also a true statement as proven previously. Then this statement provides a way to verify if a given integer $n>1$ is prime or not. If for any prime $p$, $p > \sqrt{n}$ or $p \nmid n$, then $n$ is a prime number. 
            
        \end{enumerate}

    \newpage
    \item Section 5.1 \#79. Prove that if \( p \) is a prime number and \( r \) is an integer with \( 0 < r < p \), then \( {p \choose r} \) is divisible by \( p \).
        \[
        \forall p,r \in \mathbb{R} \left( p \in \mathbb{P} \wedge r \in \mathbb{Z} : 0 < r < p \rightarrow  p \left| {p \choose r} \right. \right)
        \]
        
        \begin{proof}
        Let \( \mathbb{P} \) denote the set of all prime numbers. Let \( p \in \mathbb{P} \) and \( r \in \mathbb{Z} \) such that \( 0 < r < p \). By definition of combination, we have
        \[
        {p \choose r} = \frac{p!}{r!(p-r)!}
        \]
        By definition of factorial, it can be rewritten as
        \[
        {p \choose r} = \frac{p(p-1)!}{r!(p-r)!}
        \]
        and with some algebra
        \[
         {p \choose r} = p\paren{\frac{(p-1)!}{r!(p-r)!}}
        \]
        since $0<r<p$, all preceding term of $(p-1)!$ until $(p-r)$ can be cancelled with the denominator, leaving
        \[
        {p \choose r} = p\paren{\frac{(p-1)(p-2)\ldots (p-r+1)(p-r)!}{r!(p-r)!}} = 
        p\paren{\frac{(p-1)(p-2)\ldots (p-r+1)}{r!}}
        \]
        since $0<r<p$ and $p$ is a prime number, $r!$ does not contain the term $p$ and cannot contain a factor of $p$, hence this expression can no longer be simplified. And we can rewrite it as
        \[
        {p \choose r} = p {p-1 \choose r}
        \]
        by definition of combinations, ${p \choose r}$ has to be an integer if $p,r \in \Z$. and ${p-1 \choose r}$ also has to be an integer. Therefore
        \[
        {p \choose r} = pq_0
        \]
        where $q_0 = {p-1 \choose r} $ and $q_0 \in \Z$, and this implies that $ p \left| {p \choose r} \right.$
\end{proof}



        
\end{enumerate}


\end{document}