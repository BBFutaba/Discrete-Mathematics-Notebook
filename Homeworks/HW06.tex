\documentclass[12pt]{article}
\pagestyle{empty}
\usepackage{amsmath, amssymb, amsthm}
\usepackage{latexsym, epsfig, ulem, cancel, multicol, hyperref}
\usepackage{graphicx, tikz, subfigure,pgfplots}
\usepackage[margin=1in]{geometry}
\setlength{\parindent}{0pt}
\usepackage{multirow}
\usepackage{mathtools}


\usepackage{verbatim}
\usepackage{tikz}
\usepackage{pgfplots}


\newcommand{\T}[0]{\top}
\newcommand{\F}[0]{\bot}
\newcommand{\liminfty}[1]{\lim_{#1 \to \infty}}
\newcommand{\limzero}[1]{\lim_{#1 \to 0}}
\newcommand{\Z}{\mathbb{Z}}
\newcommand{\R}{\mathbb{R}}
\newcommand{\C}{\mathbb{C}}
\newcommand{\Q}{\mathbb{Q}}
\newcommand{\odd}[0]{\mathbb{Z} - 2\mathbb{Z}}
\newcommand{\lineint}[1]{\int_{#1}}
\newcommand{\pypx}[2]{\frac{\partial #1}{\partial #2}}
\newcommand{\divg}{\nabla \cdot}
\newcommand{\curl}{\nabla \times}
\newcommand{\dydx}[2]{\frac{d #1}{d #2}}
\newcommand{\sqbkt}[1]{\left[ #1 \right]}
\newcommand{\paren}[1]{\left( #1 \right)}
\newcommand{\tribkt}[1]{\left< #1 \right>}
\newcommand{\abso}[1]{\left|#1 \right|}
\newcommand{\zero}{\{0\}}
\newcommand{\then}{\rightarrow}
\newcommand{\nonneg}{\Z^+ \cup \{0\}}
\DeclarePairedDelimiter\ceil{\lceil}{\rceil}
\DeclarePairedDelimiter\floor{\lfloor}{\rfloor}
\newcommand{\union}[2]{\bigcup_{#1}^{#2}}
\newcommand{\inter}[2]{\bigcap_{#1}^{#2}}
\newcommand{\openclose}[1]{\left( #1 \right]}
\newcommand{\closeopen}[1]{\left[ #1 \right)}

\newcommand{\defcomp}{\exists r,s\in \Z^+ \paren{n=rs \wedge \paren{1<r<n} \wedge \paren{1<s<n}}}
\newcommand{\defprime}{\forall r,s \in \Z ^+ \paren{n=rs \rightarrow \paren{r = 1 \wedge s = n}\veebar \paren{r=n \wedge s=1}}}

\newcommand{\wsnumber}{1}
\newcommand{\wstopic}{Vectors}
\pgfplotsset{
    every linear axis/.append style={
       axis x line=center,
       axis y line=center,
       xlabel={$x$},
       ylabel={$y$}
    },
    every axis plot/.append style={thick,mark=none}
}
\tikzset{
    point/.style={circle,draw,fill,minimum width=0.3ex,inner sep=0pt,outer sep=0pt},
    every label/.append style={black}
}


\usepackage[margin=1in]{geometry}
\usepackage{amsmath, amssymb, amsthm, graphicx, hyperref}
\usepackage{enumerate}
\usepackage{fancyhdr}
\usepackage{multirow, multicol}
\usepackage{tikz}
\pagestyle{fancy}
\fancyhead[RO]{Dennis Li}
\fancyhead[LO]{Summer 6W2 2024 MA-UY 2314}
\usepackage{comment}
\newif\ifshow
\showfalse

\ifshow
  \newenvironment{solution}{\textbf{Solution.}}{}
\else
  \excludecomment{solution}
\fi

\renewcommand{\thefootnote}{\fnsymbol{footnote}}
\usepackage{comment}


\newtheorem*{remark}{Remark}


\begin{document}

\begin{center}
\ifshow
  \textbf{\Large Homework 2 Solution}\\
\else
  \textbf{\Large Homework 6}\\
\fi
Due: Tuesday July 13\\via Gradescope\\
\end{center}

\hrule

\vspace{0.2cm}

\begin{enumerate}[$\bullet$]  
\item Late homework is not accepted.  Lateness due to technical issues will not be excused.  
\end{enumerate}

\hrule

\vspace{0.5cm}



\begin{enumerate}

\item Section 7.3 \#8, 10
    \begin{enumerate}
        \item[8]
    \end{enumerate}


\item draft
    \begin{proof}
        let $x \in A \cap B$, so $x \in P(A \cap B)$ by definition of power set. Also $x \in A \wedge x \in B$ by definition of intersection. $x \in A$ by specialization, so $x \in P(A)$. similarly, $x \in P(B)$, so $x \in P(A) \cap P(B)$. And $P(A \cap B) \subseteq P(A)\cap P(B)$
        \[
        x \in A\cap B \iff x \in P(A \cap B)
        \]
        \[
        1 \neq 2 \neq 1
        \]
    \end{proof}
\item draft 2
    \begin{proof}
        let $f \colon x \to y$ and $X = \{1,2,3\}$ and $Y = \{1,2\}$. define $f = \{(1,1),(1,2),(1,3)\}$. Define $A = \{1,2\}$. We have $f(A) = \{1\}$, $f^{-1}\paren{\{1\}} = \{1,2,3\} \neq A$
    \end{proof}


\end{enumerate}


















\end{document}