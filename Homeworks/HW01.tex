\documentclass[12pt]{article}
\pagestyle{empty}
\usepackage{amsmath, amssymb, amsthm}
\usepackage{latexsym, epsfig, ulem, cancel, multicol, hyperref}
\usepackage{graphicx, tikz, subfigure,pgfplots}
\usepackage[margin=1in]{geometry}
\setlength{\parindent}{0pt}
\usepackage{multirow}
\usepackage{mathtools}


\newcommand{\R}{\mathbb{R}}
\newcommand{\dydx}{\frac{dy}{dx}}
\usepackage{verbatim}
\usepackage{tikz}
\usepackage{pgfplots}

\newcommand{\wsnumber}{1}
\newcommand{\wstopic}{Vectors}
\pgfplotsset{
    every linear axis/.append style={
       axis x line=center,
       axis y line=center,
       xlabel={$x$},
       ylabel={$y$}
    },
    every axis plot/.append style={thick,mark=none}
}
\tikzset{
    point/.style={circle,draw,fill,minimum width=0.3ex,inner sep=0pt,outer sep=0pt},
    every label/.append style={black}
}


\usepackage[margin=1in]{geometry}
\usepackage{amsmath, amssymb, amsthm, graphicx, hyperref}
\usepackage{enumerate}
\usepackage{fancyhdr}
\usepackage{multirow, multicol}
\usepackage{tikz}
\pagestyle{fancy}
\fancyhead[RO]{Dennis Li}
\fancyhead[LO]{Summer 6W2 2024 MA-UY 2314}
\usepackage{comment}
\newif\ifshow
\showfalse

\ifshow
  \newenvironment{solution}{\textbf{Solution.}}{}
\else
  \excludecomment{solution}
\fi

\renewcommand{\thefootnote}{\fnsymbol{footnote}}
\usepackage{comment}

\newcommand{\T}[0]{\top}
\newcommand{\F}[0]{\bot}

\newtheorem*{remark}{Remark}


\begin{document}

\begin{center}
\ifshow
  \textbf{\Large Homework 1 Solution}\\
\else
  \textbf{\Large Homework 1}\\
\fi
Due: Tuesday July 9\\via Gradescope\\
\end{center}

\hrule

\vspace{0.2cm}

\begin{enumerate}[$\bullet$]  
\item Late homework is not accepted.  Lateness due to technical issues will not be excused.  
\end{enumerate}

\hrule

\vspace{0.5cm}



\begin{enumerate}

    \item Show the logical equivalences using truth tables and say a few words explaining why your truth table shows\\
        \begin{enumerate}[(a)]
            \item $\neg (p \vee q) \equiv  \neg p \wedge \neg q$ (DeMorgan's Law)\\
                \[ %centers the table.
                    \begin{tabular}{|c|c|c|c|c|c|c|} 
                    \hline %\hline gives us a horizontal line.  
                    $p$ & $q$ & $\neg p$ & $\neg q$ & $p\vee q$ & $\neg (p \vee q) $ & $\neg p \wedge \neg q$  \\ %the syntax \\ ends the row.
                    \hline
                    T & T & F & F & T & \textbf{F} & \textbf{F}\\
                    \hline 
                    T & F & F & T & T & \textbf{F} & \textbf{F}\\
                    \hline
                    F & T & T & F & T & \textbf{F} & \textbf{F}\\
                    \hline
                    F & F & T & T & F & \textbf{T} & \textbf{T}\\
                    \hline
                    \end{tabular} %end of the table.  
                \] %need to close of the center of the table.  
                From the table it can be observed that the column indicating $\neg (p \vee q)$ is identical to the column indicating $\neg p \wedge \neg q $
                \[
                \therefore \neg (p \vee q) \equiv  \neg p \wedge \neg q
                \]
            \item $p\vee (q \wedge r) \equiv (p\vee q)\wedge(p \vee r)$
                \[
                    \begin{tabular}{|c|c|c|c|c|c|c|c|}
                    \hline
                    $p$ & $q$ & $r$ & $q \wedge r$ & $p \vee (q \wedge r)$ & $p \vee q$ & $p \vee r$ & $(p \vee q) \wedge (p \vee r)$\\
                    \hline
                    T&T&T&T&\textbf{T}&T&T&\textbf{T}\\
                    \hline
                    T&T&F&F&\textbf{T}&T&T&\textbf{T}\\
                    \hline
                    T&F&T&F&\textbf{T}&T&T&\textbf{T}\\
                    \hline
                    T&F&F&F&\textbf{T}&T&T&\textbf{T}\\
                    \hline
                    F&T&T&T&\textbf{T}&T&T&\textbf{T}\\
                    \hline
                    F&T&F&F&\textbf{F}&T&F&\textbf{F}\\
                    \hline
                    F&F&T&F&\textbf{F}&F&T&\textbf{F}\\
                    \hline
                    F&F&F&F&\textbf{F}&F&F&\textbf{F}\\
                    \hline
                    \end{tabular}
                \]
                From the table it can be observed that the column indicating $p \vee (q \wedge r)$ shares the identical value as the column indicating $(p \vee q) \wedge (p \vee r)$
                \[
                \therefore p \vee (q \wedge r) \equiv (p \vee q) \wedge (p \vee r)
                \]
        \end{enumerate}
        \newpage
        \item Prove that $(p \vee q) \rightarrow r \equiv (p \rightarrow r) \wedge (q \rightarrow r)$ using Theorem * and Theorem 2.1.1. Annotate your proof. For reference, see example 2.1.14.\\ \\
        \textbf{Remark:}\\
            \textit{Theorem *}\\
            \textit{Let $p$ and $q$ be statement variables, then}
            \[
            p \rightarrow q \equiv \neg p \vee q
            \]
        \begin{proof}
            \begin{align*}
                \text{let } & \text{$p$ and $q$ be statement variables}\\
                &(p \vee q) \rightarrow r  \\
                & \equiv \neg (p \vee q) \vee r \;\;\; &\text{Theorem *} \\
                & \equiv (\neg p \wedge \neg q) \vee r \;\;\;  &\text{DeMorgan's Law}  \\
                & \equiv (\neg p \vee r) \wedge (\neg q \vee r) \;\;\; &\text{Distributivity} \\
                & \equiv (p \rightarrow r) \wedge (q \rightarrow r) \;\;\; &\text{Theorem *}\\
            \end{align*}
            \[
            \therefore (p \vee q) \rightarrow r \equiv (p \rightarrow r) \wedge (q \rightarrow r)
            \]
        \end{proof}
    \newpage
    \item Find all values of $p$ and $q$ for which $p \rightarrow q$ is not equal to $q \rightarrow p$. For which values of $p$, $q$ are the statement forms equal?\\
        \[
            \begin{tabular}{|c|c|c|c|}
            \hline
            $p$ & $q$ & $p \rightarrow q$ & $q \rightarrow p$ \\
            \hline
                 T&T&T&T  \\
            \hline
                 T&F&F&T\\
            \hline
                 F&T&T&F\\
            \hline
                 F&F&T&T \\
            \hline
            \end{tabular}
        \]
        From the table it can be seen that when both $p$ and $q$ have the same truth value, two statements equivalent to each other. And when $p$ and $q$ do not have the same truth value, two statements are not equivalent to each other.
        \newpage
    \item Show that $[(p \rightarrow q) \wedge (p \rightarrow \neg q)] \rightarrow \neg p$ is a tautology using Theorem * and Theorem 2.1.1. Annotate your proof.
        \begin{proof}
            \begin{align*}
                \text{let } & \text{$p$, $q$, and $r$ be statement variables and $r \equiv \neg p$}\\
                & [(p \rightarrow q) \wedge (p \rightarrow \neg q)] \rightarrow \neg p  \\
                \equiv &  [(p \rightarrow q) \wedge (p \rightarrow \neg q)] \rightarrow r & \text{Substitution}\\
                \equiv & [(\neg p \vee q) \wedge (\neg p \vee \neg q)] \rightarrow r & \text{Theorem *}\\
                \equiv & [(r \vee q) \wedge (r \vee \neg q)] \rightarrow r & \text{Substitution} \\
                \equiv & [ r \wedge (q \vee \neg q )] \rightarrow r & \text{Distributivity}\\
                \equiv & [r \wedge \T ] \rightarrow r & \text{Negation Law}\\
                \equiv & r \rightarrow r & \text{Identity Law}\\
                \equiv & \neg r \vee r & \text{Theorem *}\\
                \equiv &  r \vee \neg r & \text{Commutativity}\\
                \equiv & \T & \text{Negation Law}
            \end{align*}
            \[
            \therefore [(p \rightarrow q) \wedge (p \rightarrow \neg q)] \rightarrow \neg p \equiv \T
            \]
        \end{proof}
    \newpage
    \item Let $s$ be a string of length 2 with characters from \{0,1,2\}, and define statements $a,b,c$ and $d$ as follows:\\
        \[
            \begin{cases}
                a = \text{the first character of $s$ is 0}\\
                b = \text{the first character of $s$ is 1}\\
                c = \text{the second character of $s$ is 1}\\
                d = \text{the second character of $s$ is 2}
            \end{cases}
        \]
        \begin{enumerate}[a.]
            \item $(a\vee b)\wedge (c\vee d)$\\
                For [a.] to be true, $(a\vee b)$ has to be true and $(c\vee d)$ has to be true. For $(a\vee b)$ to be true, statement $a$ and $b$ cannot be false at the same time. Therefore the first item in $s$ cannot be $2$. The same argument applies to $(c\vee d)$. Therefore the second item in $s$ cannot be 0.\\
                The statement [a.] will hold true for all combinations of $s$ except for $\{2,0\}$.
            \\
            \item $\left[\neg (a\vee b)\right] \wedge (c \vee d)$\\
                For [b.] to be true, $\left[\neg (a\vee b)\right]$ and $(c \vee d)$ have to be true. For $\left[\neg (a\vee b)\right]$ to be true, $(a\vee b)$ has to be false, and as explained in the previous segment, the first item in the string has to be $2$. $(c\vee d)$ is true if the second item in the string is not $0$.\\
                The statement [b.] will hold true for all combinations of $s$ that has $2$ as the first item and anything but $0$ at the second item.\\
            \item $(\neg a \vee b)\wedge (c \vee \neg d)$
                For [c.] to be true, $(\neg a \vee b)$ and $(c \vee \neg d)$ has to be true simultaneously. For $(\neg a \vee b)$ to be true, the first item in the $s$ has to be 1. For $(c \vee \neg d)$ to be true, the second item of $s$ has to be 1 as well.\\
                The statement [c.] will hold true for $s = \{1,1\}$ only. 
            
        \end{enumerate}
        \newpage
    \item Let the symbol $\oplus$ denote \textit{exclusive or}; and $ p \oplus q \equiv (p \vee q ) \wedge \neg (p \wedge q)$. Its truth table is as follows:
        \[
            \begin{array}{|c|c|c|}
            \hline
            p & q & p \oplus q \\
            \hline
            \text{T} & \text{T} & \text{F} \\
            \hline
            \text{T} & \text{F} & \text{T} \\
            \hline
            \text{F} & \text{T} & \text{T} \\
            \hline
            \text{F} & \text{F} & \text{F} \\
            \hline
            \end{array}
        \]
        Find a simpler statement forms that are logically equivalent to $p \oplus p$ and $(p \oplus p) \oplus p$\\
        \begin{proof}[Proof 1]
        
            \begin{align*}
            \text{let } & \text{$p$ be a statement variable}\\
                & p \oplus p \\
                \equiv & (p \vee p ) \wedge \neg (p \wedge p) & \text{By Definition}\\
                \equiv & p \wedge \neg p & \text{Idempotence} \\
                \equiv & \F & \text{Negation Law}
            \end{align*}
        \end{proof}
        \begin{proof}[Proof 2]
            \[
            \therefore p \oplus p \equiv \F
            \]
            \begin{align*}
                & (p \oplus p) \oplus p \\
                \equiv & \F \oplus p & \text{by Proof 1}\\
                \equiv & (\F \vee p ) \wedge \neg (\F \wedge p) & \text{By Definition}\\
                \equiv & (p \vee \F ) \wedge \neg ( p \wedge \F ) & \text{Commutativity}\\
                \equiv & p \wedge \neg ( p \wedge \F) & \text{Identity Law} \\
                \equiv & p \wedge \neg \F & \text{Universal Bound}\\
                \equiv & p \wedge \T & \text{Contradiction Negation}\\
                \equiv & p & \text{Identity Law}
            \end{align*}
            \[
            \therefore (p \oplus p) \oplus p \equiv p
            \]
        \end{proof}
    \newpage
    \item For the following statements, write its contrapositives, converse, and inverse.
        \begin{enumerate}
            \item[b.] If today is New Year's Eve, then tomorrow is January.
                \begin{enumerate}
                    \item[Contrapositive:] If tomorrow is not January, then today is not New Year's Eve. 
                    \item[Converse:] If tomorrow is January, then today is New Year's Eve.
                    \item[Inverse:] If today is not New Year's Eve, then tomorrow is not January.
                    \\
                \end{enumerate}
            \item[c.] If the decimal expansion of $r$ is terminating, then $r$ is rational
                \begin{enumerate}
                    \item[Contrapositive:] If $r$ is not rational, then the decimal expansion of $r$ is not terminating.
                    \item[Converse:] If $r$ is rational, then the decimal expansion of $r$ is terminating.
                    \item[Inverse:] If the decimal expansion of $r$ is not terminating, then $r$ is not rational.
                    \\
                \end{enumerate}
            \item[e.] If $x$ is non-negative, then $x$ is positive or $x$ is 0
                \begin{enumerate}
                    \item[Contrapositive:] If $x$ is not positive and $x$ is not 0, then $x$ is not non-negative.
                    \item[Converse:] If $x$ is positive or $x$ is 0, then $x$ is non-negative.
                    \item[Inverse:]  If $x$ is not non-negative, then $x$ is not positive and $x$ is not 0.
                    \\
                \end{enumerate}
            \item[g.] If $n$ is divisible by 6, then $n$ is divisible by 2 and $n$ is divisible by 3.
                \begin{enumerate}
                    \item[Contrapositive:] If $n$ is not divisible by 2 or $3$, then $n$ is not divisible by 6.
                    \item[Converse:] If $n$ is divisible by 2 and 3, then $n$ is divisible by 6.
                    \item[Inverse:] If $n$ is not divisible by 6, then $n$ is not divisible by 2 or 3. 
                \end{enumerate}
        \end{enumerate}
        \newpage

    \item Exercise 14,38,43
        \begin{enumerate}
            \item[14.a] Show that the following statement forms are all logically equivalent:
            \[
            \begin{cases}
                p \rightarrow q\vee r\\
                p \wedge \neg q \rightarrow r\\
                p\wedge \neg r \rightarrow q
            \end{cases}
            \]
            \begin{proof}
                \begin{align*}
                    & p \rightarrow q\vee r\\
                    \equiv & p \rightarrow (q\vee r)\\
                    \equiv & \neg p \vee (q \vee r) & \text{Theorem *} \\
                    \equiv & \neg p \vee q \vee r
                \end{align*}
                \begin{align*}
                    & p \wedge \neg q \rightarrow r\\
                    \equiv & (p \wedge \neg q) \rightarrow r\\
                    \equiv & \neg (p \wedge \neg q) \vee r & \text{Theorem *} \\
                    \equiv & \neg p \vee \neg \neg q \vee r & \text{DeMorgan's Law} \\
                    \equiv & \neg p \vee q \vee r & \text{Double Negative Law} \\
                \end{align*}
                \begin{align*}
                    & p\wedge \neg r \rightarrow q\\
                    \equiv & (p\wedge \neg r) \rightarrow q\\
                    \equiv & \neg (p\wedge \neg r) \vee q & \text{Theorem *}\\
                    \equiv & \neg p \vee \neg \neg r \vee q & \text{DeMorgan's Law}\\
                    \equiv & \neg p \vee r \vee q & \text{Double Negative Law}\\
                    \equiv & \neg p \vee q \vee r & \text{Commutativity}
                \end{align*}
                \begin{align*}
                \therefore\;\; & p \rightarrow q\vee r \\
                \equiv & p \wedge \neg q \rightarrow r\\
                \equiv & p\wedge \neg r \rightarrow q
                \end{align*}
            \end{proof}

            \item[14.b] Use the established equivalence, rewrite the following sentence.
            \[
            \text{If $n$ is prime, then $n$ is odd or $n$ is 2.}
            \]
            The original sentence corresponds to $p \rightarrow q\vee r$.\\
            Case 2: $p \wedge \neg q \rightarrow r$
            \[
            \text{If $n$ is prime and $n$ is not odd, then $n$ is 2.}
            \]
            Case 3: $p\wedge \neg r \rightarrow q$
            \[
            \text{If $n$ is prime and $n$ is not 2, then $n$ is odd.}
            \]

        \newpage
        \item[38.] Rewrite the following statement in \textit{if~then} form.
        \[
        \text{Ann will go unless it rains.}
        \]
        Recall: $r$ unless $s$ means if $\neg s$ then $r$
        \[
        \text{If it does not rain, Ann will go.}
        \]

        \item[43.] Use the contrapositive to rewrite the statement in 43 in if-then form in two ways.
        \[
        \text{Doing homework regularly is a necessary condition for Jim to pass the course.}
        \]
        Rewrites into
        \[
        \text{If Jim does not do homework regularly, Jim cannot pass the course.}
        \]
        \end{enumerate}






\end{enumerate}


\end{document}